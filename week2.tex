We have the following important theorem about closed forms on the plane (or subsets thereof). 
\begin{theorem}\label{thm:integral-homotopic-curves}
Let $\omega$ be a closed differential form in open $\Omega \subset \R^2$. If $\gamma_0, \gamma_1: [0, 1] \to \Omega$ homotopic curves (either with fixed endpoints or as closed curves) then
$$\int_{\gamma_1} \omega = \int_{\gamma_2} \omega$$
\end{theorem}

From this, \autoref{thm:global-prim-in-simp-connected} follows as a corollary since it immediately implies that the integral over any closed curve is 0.
\begin{theorem}\label{thm:global-prim-in-simp-connected}
    A closed differential form $\omega$ in a simply connected open set $\Omega \subset \R^2$ has a global primitive.
\end{theorem}

\subsection{Cauchy's Integral Formula}
We are very close to being able to state and prove Cauchy's Integral Formula. The one thing that is left is the winding number.
\begin{definition}
The winding number of a closed curve $\gamma$ with respect to a point $a$ (not on $\gamma$) is given by
$$w(\gamma, a) := \frac{1}{2\pi i} \int_\gamma \frac{1}{z - a}dz$$
\end{definition}

It is clear that $w(\gamma, a)$ is an integer because the integral is the difference between 2 branches of $\log$. If $\gamma$ is the boundary of a circle then
\begin{align*}
    w(\gamma, a) = 
    \begin{cases}
        1 & a \text{ inside circle}\\
        0 & a \text{ outside circle}
    \end{cases}
\end{align*}

From \autoref{thm:integral-homotopic-curves}, it follows that $w(\gamma, a)$ is invariant under homotopy of $\gamma$ that does not pass through $a$. Since moving $\gamma$ a little bit is the same as moving $a$ a little bit, it also follows that $w(\gamma, \cdot)$ is constant on the connected components of the complement of $\gamma$. 

\begin{theorem}
    Suppose $f(z)$ is holomorphic in an open $\Omega \subset \C$ and $a$ is a point in $\Omega$. Let $\gamma$ be a nullhomotopic closed curve in $\Omega$. Then
    $$\frac{1}{2\pi i}\int_\gamma \frac{f(z)}{z - a} dz = f(a) w(\gamma, a)$$
\end{theorem}
\begin{proof}
    We define
    $$g(z) = \begin{cases}
        \frac{f(z) - f(a)}{z - a} & z \neq a\\
        f'(a) & z = a
    \end{cases}$$
    We see that $g$ is continuous in $\Omega$ and holomorphic on $\Omega \setminus \{a\}$. Therefore $g(z) dz$ is closed by Cauchy's Theorem. Then the nullhomotopy of $\gamma$ implies that
    $$\int_\gamma \frac{f(z) - f(a)}{z - a} dz = 0$$
    Splitting the sum gets us the desired result
    $$\int_\gamma \frac{f(z)}{z - a} dz = \int_\gamma \frac{f(a)}{z - a} dz = 2\pi i f(a) w(\gamma, a)$$
    
\end{proof}

A very nice and important consequence of Cauchy's integral formula is that holomorphic functions are infinitely differentiable. Suppose $f$ is a holomorphic function in a neighbourhood of a closed disk $\abs{z} \leq r$. For $\abs{z} < r$, we have
$$f(z) = \frac{1}{2\pi i}\int_\gamma \frac{f(\zeta)}{\zeta - z} d\zeta$$
Then we can differentiate both sides (by differentiating under the integral sign) to get
$$f'(z) = \frac{1}{2\pi i}\int_\gamma \frac{f(\zeta)}{(\zeta - z)^2} d\zeta$$
and more generally
$$f^{(n)}(z) = \frac{n!}{2\pi i} \int_\gamma \frac{f(\zeta)}{(\zeta - z)^{n + 1}} d\zeta$$

We can summarise all this information about holomorphic function as follows.
\begin{theorem} \label{thm:holom-tfae}
    Suppose $f(z)$ is a continuous function on an open set $\Omega$. Then the following are equivalent:
    \begin{enumerate}[label=\arabic*)]
        \item $f(z)$ is holomorphic
        \item $f(z) dz$ is closed
        \item Given $\gamma$ the boundary of a circle of radius $r$ and $\abs{z} < r$, we have $$f(z) = \frac{1}{2\pi i}\int_\gamma \frac{f(\zeta)}{\zeta - z} d\zeta$$ 
    \end{enumerate}
\end{theorem}
\begin{proof}
    We know that 1) $\Rightarrow$ 2) is Cauchy's theorem. We have shown 1) $\Rightarrow$ 3) above. 3) $\Rightarrow$ 1) is easy to see since we can differentiate under the integral sign. This only leaves 2) $\Rightarrow$ 1) which is known as Morera's theorem.

    If $f(z) dz$ is closed then $f(z) dz$ locally has a primitive $g(z)$. Then
    $$f(z)dz = dg = \frac{\partial g}{\partial z}dz + \frac{\partial g}{\partial \ol{z}} d\ol{z}$$
    Since $dz$ and $d\ol{z}$ are linearly independent, we must have $\frac{\partial g}{\partial \ol{z}} = 0$. This also tells us that $f(z) = \frac{\partial g}{\partial z}$ and hence $f$ is holomorphic since the derivative of holomorphic functions is holomorphic.
\end{proof}

Not only are holomorphic functions infinitely differentiable, they are also analytic, which is to say every holomorphic function has a convergent power series expansion (defined locally of course) that represents the function.

By Cauchy's Integral formula we know that if $f(z)$ is a holomorphic function in a neighbourhood of $\abs{z} \leq r$, then
$$f(z) = \int_\gamma \frac{f(\zeta)}{\zeta - z} d\zeta$$
where $\gamma$ is the boundary of the closed disk, $\abs{z} = r$.
We can then write
\begin{align*}
    \frac{1}{\zeta - z} = \frac{1}{\zeta} \left( 1 - \frac{z}{\zeta} \right)^{-1} = \frac{1}{\zeta} \sum_{n = 0}^\infty \frac{z^n}{\zeta^n}
\end{align*}
We can substitute this into the integral formula to get
$$ f(z) = \sum_{n = 0}^\infty \left( \frac{1}{2\pi i} \int_\gamma \frac{f(\zeta)}{\zeta^{n + 1}} d\zeta \right) z^n $$
The coefficient of $z^n$ agrees with what we would expect since we know that $a_n = \frac{f^{(n)}(0)}{n!}$. This series converges whenever $\abs{z} < r$.

We can also bound the Taylor coefficients. For example by substituting $z = re^{i \theta}$ we get
$$f(re^{i \theta}) = \sum_{m = 0}^\infty a_m r^m e^{im \theta}$$
We can multiply both sides by $e^{-in \theta}$ and integrate from $\theta = 0$ to $\theta = 2\pi$. This will cancel out all but the $a_n r^n e^{i n \theta}$ term in the series. Hence we get
$$a_n r^n = \frac{1}{2\pi} \int_0^{2\pi} f(re^{i \theta}) e^{-in \theta} d\theta$$
If we have $M(r) := \sup_{\abs{z} = r} \abs{f(z)}$ then
$$\abs{a_n} \leq \frac{M(r)}{r^n}$$
These are known as \textit{Cauchy's inequalities}.

This immediately gives us Liouville's Theorem (a corollary of which is the Fundamental Theorem of Algebra).
\begin{theorem}[Liouville's Theorem]
    If $f(z)$ is holomorphic on $\C$ and bounded then $f$ is constant.
\end{theorem}
\begin{proof}
    There is some $M$ such that $M(r) \leq M$ for all $r$. Therefore
    $$\abs{a_n} \leq \frac{M}{r^n}$$
    Therefore for $n \geq 1$, we can send $r \to \infty$ to conclude that $a_n = 0$. Therefore $f(z) = a_0$ is constant.
\end{proof}

A direct consequence of the integral formula is that holomorphic functions satisfy the Mean Value Property which is to say that given a closed disk $D$ of radius $r$, centered at a point $a$, the value of $f(a)$ is given by the mean value along the boundary of the disk. In formulae, we write
$$f(a) = \frac{1}{2\pi} \int_{0}^{2 \pi} f(a + re^{i \theta}) d\theta $$

Continuous functions that have the mean value property also satisfy the very important maximum modulus principle.
\begin{theorem}[Maximum Modulus Principle]\label{thm:max-mod-prin}
    Suppose $f(z)$ is a continuous complex-valued function defined on an open set $\Omega$ and $f$ satisfies the Mean Value Property. If $\abs{f}$ has a local max at $z_0 \in \Omega$, then $f$ is constant in a neighbourhood of $z_0$.
\end{theorem}

\subsection{Harmonic Functions, revisited}\label{subsec:harmonic-func-revisit}
Earlier it was claimed that all real-valued harmonic functions are (locally) the real part of a holomorphic function and moreover this holomorphic function is unique up to the addition of a constant. Let us verify this claim. Let $g$ be a real-valued harmonic function. Then
$$\frac{\partial^2 g}{\partial z \partial{\ol{z}}} = 0$$
This means that $\frac{\partial g}{\partial z}$ is holomorphic and therefore $\frac{\partial g}{\partial z}dz$ locally has a holomorphic primitive. Let us call this primitive $f$. Then
$$df = \frac{\partial g}{\partial z}dz$$
Conjugating both sides we get
$$d \ol{f} = \frac{\partial g}{\partial \ol{z}} d\ol{z}$$
where $g$ is not conjugated since it is real valued. Therefore we have
$$df + d \ol{f} = d(f + \ol{f}) = \frac{\partial g}{\partial z}dz + \frac{\partial g}{\partial \ol{z}} d\ol{z} = dg$$
Therefore $g = f + \ol{f}$ up to the addition of a constant.

It is clear that if $f(z)$ is a complex-valued functions with the Mean Value Property then the real and imaginary parts of $f$ have this property (we can simply equate the real and imaginary parts on both sides). Since harmonic functions are locally the real part of a holomorphic function, it follows that harmonic functions also satisfy the Mean Value Property. In fact any continuous function satisfying the Mean Value Property is harmonic, which we will soon prove.

The natural question that arises now is whether given a real-valued harmonic function we can figure out what the corresponding holomorphic functions should be. Suppose the harmonic function is $g(z)$ and the holomorphic function is $f(z)$. We know there for some $R$ and any $\abs{z} < R$ we have
\begin{align*}
    f(z) = \sum_{n = 0}^\infty a_n z^n
\end{align*}
Moreover since $f$ is unique up to addition of a constant, we can assume that $f(0) = a_0$ is real. Then substituting $z = re^{i \theta}$ and equating the real part we get
$$g(r \cos \theta, r \sin \theta) = a_0 + \frac{1}{2}\sum_{n = 0}^\infty (a_n r^n e^{in \theta} + \ol{a_n} r^n e^{-in \theta})$$
Therefore we get that
$$a_0 = \frac{1}{2\pi} \int_0^{2\pi} g(r \cos \theta, r \sin \theta) d\theta$$
For the remaining coefficients, we can use our usual trick of multiplying by $e^{-in \theta}$ to conclude
$$a_n = \frac{1}{\pi} \int_0^{2\pi} g(r \cos \theta, r \sin \theta) \frac{1}{r^n e^{in \theta}} d\theta$$
Substituting these back into the expansion of $f$ we get
$$f(z) = \sum_{n = 0}^\infty a_n z^n = \frac{1}{2\pi} \int_0^{2 \pi} g(r \cos \theta, r \sin \theta) \left[ 1 + 2 \sum_{n = 0}^\infty \left( \frac{z}{re^{i \theta}} \right)^n \right] d\theta$$
We can evaluate the series and simplify things to get
$$f(z) = \frac{1}{2\pi} \int_0^{2\pi} g(r \cos \theta, r \sin \theta) \frac{re^{i \theta} + z}{re^{i \theta} - z} d\theta$$
Equating real parts again we get
$$g(z) = \frac{1}{2\pi} \int_0^{2\pi} g(r \cos \theta, r \sin \theta) \frac{r^2 - \abs{z}^2}{\abs{re^{i \theta} - z}^2} d\theta$$
\begin{remark}
We call the function
$$\frac{r^2 - \abs{z}^2}{\abs{re^{i \theta} - z}^2}$$
the Poisson kernel.
\end{remark}

A classic problem with harmonic functions is the Dirichlet problem. In this case we work on the disk. 
\begin{theorem}
    Given a continuous function $f(\theta)$ which is periodic and has period $2\pi$ and given some $r > 0$, there exists a continuous function $F(z)$ on the closed disk $\abs{z} \leq r$ which is harmonic on the open disk (of radius $r$) with $F(re^{i \theta}) = f(\theta)$. Moreover $F$ is unique.
\end{theorem}
\begin{proof}
    We can assume that $f$ is real-valued (otherwise we work with the real and imaginary parts of $f$ separately).

    The uniqueness of $F$ follows from the maximum modulus principle. For existence we can define
    $$F(z) = \frac{1}{2\pi} \int_0^{2\pi} f(\theta) \frac{r^2 - \abs{z}^2}{\abs{re^{i \theta} - z}^2} d\theta$$
    which is harmonic because it is the real part of the holomorphic function
    $$ \frac{1}{2\pi} \int_0^{2\pi} f(\theta) \frac{re^{i \theta} + z}{re^{i \theta} - z} d\theta $$
    All that remains to check is that $\lim_{z \to re^{i \theta_0}} F(z) = f(\theta_0)$ which is a direct computation. As usual, more details can be found in my MAT354 notes.
\end{proof}

\begin{corollary}
A continuous function $f(z)$ defined on an open $\Omega \subset \R^2$ with the Mean Value Property is harmonic.
\end{corollary}
\begin{proof} 
    It suffices to check things locally. So let $z_0$ be some point in $\Omega$ and $D$ some disk in $\Omega$ that contains $z_0$. Then there exists a function $F$ which is continuous on $\ol{D}$ and harmonic on $D$ and which agrees with $f$ on the boundary of $D$. Since $F$ and $f$ both have the Mean Value Property so does $F - f$. Note this function is 0 on the boundary of $D$ and therefore is identically 0 on $D$ by the maximum modulus principle.
\end{proof}

\subsection{Zeros, poles and singularities}
Suppose $f(z)$ is a holomorphic function such that $f(z_0) = 0$. Then near $z_0$ we can use the power series of $f$ to write
$$f(z) = (z - z_0)^k f_1(z)$$
where $f_1$ is holomorphic and non-zero at $z_0$. The integer $k$ is known as the order or multiplicity of $z_0$.

\begin{definition}[Meromorphic functions]
    A \textit{meromorphic function} on an open set $\Omega$ is a function that is holomorphic on the complement of a discrete subset of $\Omega$ and expressible in a neighbourhood of any point of $\Omega$ as the quotient of holomorphic functions $\frac{f(z)}{g(z)}$ (where of course $g$ is not identically 0).
\end{definition}

If $f(z)$ and $g(z)$ are holomorphic functions, we can write
\begin{align*}
    f(z) &= (z - z_0)^k f_1(z)\\
    g(z) &= (z - z_0)^l g_1(z)
\end{align*}
where $f_1$ and $g_1$ are both non-zero at $z_0$. Then
$$ \left( \frac{f}{g} \right)(z) = (z - z_0)^{k - l} \left( \frac{f_1}{g_1} \right)(z)$$

If $k \geq l$, then $f/g$ extends holomorphically at $z_0$. Otherwise we have $\lim_{z \to z_0} (\frac{f}{g})(z) = \infty$, so $z_0$ is a pole of order $l - k$. The limit can be thought of as a convergence to the point at infinity in the Riemann sphere. Therefore we can also consider meromorphic functions as functions with values in $S^2$. With this we see that meromorphic functions are simply holomorphic functions with values in $S^2$.